
\documentclass[11pt,a4paper,sans]{moderncv} % Font sizes: 10, 11, or 12; paper sizes: a4paper, letterpaper, a5paper, legalpaper, executivepaper or landscape; font families: sans or roman
\usepackage[utf8]{inputenc}

\moderncvstyle{casual} % CV theme - options include: 'casual' (default), 'classic', 'oldstyle' and 'banking'
\moderncvcolor{blue} % CV color - options include: 'blue' (default), 'orange', 'green', 'red', 'purple', 'grey' and 'black'

\usepackage{lipsum} % Used for inserting dummy 'Lorem ipsum' text into the template

\usepackage[scale=0.75]{geometry} % Reduce document margins
%\setlength{\hintscolumnwidth}{3cm} % Uncomment to change the width of the dates column
%\setlength{\makecvtitlenamewidth}{10cm} % For the 'classic' style, uncomment to adjust the width of the space allocated to your name

%----------------------------------------------------------------------------------------
%	NAME AND CONTACT INFORMATION SECTION
%----------------------------------------------------------------------------------------

\firstname{Aurèle} % Your first name
\familyname{Barrière} % Your last name

% All information in this block is optional, comment out any lines you don't need
\title{Curriculum Vitae}
\mobile{06 19 40 78 60}
\email{aurele.barriere@ens-rennes.fr}
\homepage{perso.eleves.ens-rennes.fr/people/Aurele.Barriere}{perso.eleves.ens-rennes.fr/people/Aurele.Barriere} % The first argument is the url for the clickable link, the second argument is the url displayed in the template - this allows special characters to be displayed such as the tilde in this example
%\extrainfo{additional information}
%\photo[70pt][0.4pt]{pictures/picture} % The first bracket is the picture height, the second is the thickness of the frame around the picture (0pt for no frame)
%\quote{"A witty and playful quotation" - John Smith}

%----------------------------------------------------------------------------------------

\begin{document}

\makecvtitle % Print the CV title

%----------------------------------------------------------------------------------------
%	EDUCATION SECTION
%----------------------------------------------------------------------------------------

\section{Éducation}

\cventry{2016--2017}{M1RI - Magistère Informatique}{ENS Rennes}{}{}{}
\cventry{2015--2016}{L3RI - Magistère Informatique}{ENS Rennes}{}{}{}  % Arguments not required can be left empty
\cventry{2014--2015}{MP*}{Lycée Marcelin Berthelot}{Saint-Maur-des-Fossés}{}{}
\cventry{2013--2014}{MPSI}{Lycée Marcelin Berthelot}{Saint-Maur-des-Fossés}{}{}
\cventry{2012--2013}{Baccalauréat}{Lycée Christophe Colomb}{Sucy-en-Brie}{}{Mention Très Bien}

%
%   Internships
%
\section{Stage}
\cvitem{2016}{Estimation de WCET haut-niveau avec Interprétation abstraite et Programmation par Contraintes}



%----------------------------------------------------------------------------------------
%	AWARDS SECTION
%----------------------------------------------------------------------------------------

\section{Récompenses}

\cvitem{2012}{Classé $5^{eme}$ aux Olympiades Académiques de Mathématiques de l'académie de Créteil}

%----------------------------------------------------------------------------------------
%	COMPUTER SKILLS SECTION
%----------------------------------------------------------------------------------------

\section{Langages informatiques}

\cvitem{}{OCamL, \TeX/\LaTeX, Coq, C++, Python, C, Scala, Bash}



%----------------------------------------------------------------------------------------
%	LANGUAGES SECTION
%----------------------------------------------------------------------------------------

\section{Langages}

\cvitemwithcomment{Français}{Langue maternelle}{}
\cvitemwithcomment{Anglais}{Avancé}{}
%\cvitemwithcomment{Allemand}{Très basique}{}

%----------------------------------------------------------------------------------------
%	INTERESTS SECTION
%----------------------------------------------------------------------------------------

%\section{Intérêts}

\renewcommand{\listitemsymbol}{-~} % Changes the symbol used for lists

%\cvlistdoubleitem{}{}
%\cvlistitem{}

\end{document}
